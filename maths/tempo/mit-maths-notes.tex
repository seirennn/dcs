\documentclass{article}
\usepackage{amssymb, mathrsfs, enumitem}
\usepackage{tikz-cd}
\usepackage{multicol}
\usepackage{graphicx}
\usepackage{color}
\usepackage{amsthm}
\usepackage{hyperref}
\usepackage{verbatim}
\usepackage{caption}
\usepackage{subcaption}
\usepackage{amsmath, epsfig, colortbl}
\usepackage[breakable, skins]{tcolorbox}
\setlength{\parskip}{8pt}

\newtheorem{theorem}{Theorem}
\newtheorem{lemma}{Lemma}
\newtheorem{cor}{Corollary}
\theoremstyle{definition}
\newtheorem{prop}{Proposition}
\newtheorem{fact}{Fact}
\newtheorem{example}{Example}
\newtheorem{definition}{Definition}
\newenvironment{solution} {\paragraph{Solution.}}{\hfill$\square$}
\renewenvironment{proof} {\paragraph{Proof.}}{\hfill$\square$}
\renewcommand{\today}{}
\captionsetup{margin=.7in, format=hang}

\newtcolorbox{examplebox}{ breakable, enhanced}
\newtcolorbox{gptbox}{ breakable, enhanced,  colback  = blue!10}

\addtolength{\topmargin}{-.875in}
\addtolength{\oddsidemargin}{-.8in}
\addtolength{\textwidth}{2in}
\addtolength{\textheight}{2.15in}

\newcommand{\mc}{\mathcal}
\newcommand{\abs}[1]{\left| #1\right|}
\newcommand{\la}{\Leftarrow}
\newcommand{\ra}{\Rightarrow}
\renewcommand{\iff}{\Leftrightarrow}
\newcommand{\Z}{\mathbb{Z}}
\newcommand{\floor}[1]{\left\lfloor #1\right\rfloor}
\newcommand{\ceil}[1]{\left\lceil #1\right\rceil}

\newcommand{\ex}[1]{\begin{examplebox}\begin{example} #1 \end{example}\end{examplebox}}

\newcommand{\eps}{\epsilon}
\newcommand{\N}{\mathbb{N}}
\newcommand{\Q}{\mathbb{Q}}
\newcommand{\R}{\mathbb{R}}
\newcommand{\ba}{\backslash}

\def\cents{\hbox{\rm\rlap/c}}




\begin{document}
\title{MIT Fall 2010 Lecture notes: 6.042J Mathematics for Computer Science }
\author{Notes by Lanso Humtsoe}
\maketitle
\section{Introduction and Proofs}

Let us begin on these lectures notes on Mathematics for Computer science first by defining what a proof is. 

\begin{definition}
In the context of Mathematics, A proof is a logical argument that establishes the truth of a mathematical statement, in other definition, A mathematical proof of a proposition is a chain of logical deductions leading to the proposition from a base to set of axioms.
\end{definition}
 The purpose of a proof is to establish both oneself and others that a particular assetion or a preposition is     true, based on a set of axioms and previously estalished truths(Theorems). 

 Here are the key elements that are included in a mathematical proofs:

\begin{itemize}
  \item \textbf{Statement of Theorem and Proposition:} clearing stating what is to be proven.
  \item \textbf{Basis of Assumptions:} beginning with a set of accepted axioms or previously proven theorems. 
  \item \textbf{Logical deductions:} A sequence of logical steps, using accepted rules of inference, that lead from the assumptions to the conclusion.
  \item \textbf{Conclusion:}A clear and a unambiguous statement that the theorem or preposition has been proved.
\end{itemize}

But thats a Mathematical proof in the context of mathematics, and proof exist beyond mathamatics, there is a higher notion of a proof, that may have no logical deductions or assumptions.
this higher level meta notion of a proof can be defined as a method for ascertaining the truth.

now, ascertaining simply meaning establishing truth, verifying truth, and there are lots of ways to ascertain truth in society and even in science.

coming backt to the mathematical definition of proof, it is a verfification of a assertion or proposition by a chain of logical deductions set from a set of axioms. 


Let us further elaborate on propositions, logical deductions and axioms, let's start with propositions:

\subsection{Proposition}

A proposition is a statement that is either true or false.
here is a simple example.

\begin{example}
  2 + 3 = 5
\end{example}
\begin{example}
  1 + 1 = 3
\end{example}
  
here is a more interesting example...

\begin{example}
$\forall$n$\in$$\N$, $n^2$+$n$+41 is a prime number.
\end{example}
the $n^2$+$n$+41 is an example of a predicate, A Predicate is a proposition whose truth depends on the value of the variable, in this case, $n$ is that variable in the example.

let us further elaborate of the different types of propositions.

\subsection{Atomic proposition}
These are simple basic, indivisible statements that cannot be further broken down into simpler components.

\begin{example}
 2 + 2 = 4, "the sky is blue", 1 + 1 = 2.
\end{example}

\subsection{Compound proposition}
These propositions are formed using simple propisiton using logical connectives.
common logical connectives include:

\begin{itemize}
  \item \textbf{Conjunction(AND):} $p$ AND $q$ ( both $p$ and $q$ has to be true for the compound propositiob to be true).
  \item \textbf{Disjunction(OR):} $p$ OR $q$ ( either $p$ or $q$ has to be true for the Disjunction proposition to be true).
  \item \textbf{Negation(NOT):} "NOT $p$" (The opposite true value of $p$).
\end{itemize}

\subsection{Conditional proposition}
These propositions express a relationship of implications. the conditional proposition of "if $p$ then $q$" is denoted as $p$ $\to$ $q$. it is false only when $p$ is true and $q$ is false; otherwise, it is tru it is false only when $p$ is true and $q$ is false; otherwise, it is true. 
here is an example to further explain: 

\begin{example}
  $p$: "$x$ $>$ $5$"
\end{example}

\begin{example}
  $q$: "$2x > 10$ "
\end{example}

the conditional propostion is expressed as $p > q$.

\subsection{Biconditional proposition}
These propositions express bidirectional relationship. The Biconditional propistion "if $p$ and only if $q$" is denoted by $p \leftrightarrow q$. it is true when both $p$ and $q$ have same truth value.

\subsection{Existential proposition}
These proposition assert the existence of at the least one instances satisfying a given property. 

\end{document}
